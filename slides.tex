\documentclass[14pt,utf8,hyperref={pdfpagelabels=false}]{beamer}
\usepackage{etex}
\usepackage{media9}

\usepackage[utf8]{inputenc} % hyperref broken with utf8x
\usepackage[C40,T1]{fontenc}

\usepackage{graphicx}
\usepackage{array}
\usepackage{multirow}

\usepackage[francais]{babel}

\usepackage{algorithmic, algorithm}

\usepackage{tikz, pgfplots}
\usepackage{tikz-qtree}

\newcommand{\umltextcolor}{ThoughfulBrick}
\newcommand{\umldrawcolor}{Thoughtless}
\newcommand{\umlfillcolor}{ThoughtBySome}

%\reserveinserts{28}
%\usepackage{pgfpages}
%\setbeameroption{show notes on second screen}


\usetikzlibrary{shapes.arrows,chains,positioning,%
  matrix,patterns,shapes,shapes.multipart,positioning,arrows}
%\tikzexternalize

\tikzstyle{every picture}+=[remember picture]
\tikzstyle{na} = [baseline=-.5ex]

\tikzstyle{state}=[rectangle,
  color=structure.fg,
  fill=ThoughtBySome,
  inner sep=0.2cm,
  rounded corners=2mm]

\tikzstyle{mathbox} = [inner sep=0pt, anchor=base,%
  node distance=0pt, column sep=0em, thick]

\tikzset{
  template parameter/.style={
        append after command={
            node [draw, densely dashed, umlcolor, font=\ttfamily]
                at (\tikzlastnode.north east)
                {#1}
        }
    }
}

\tikzset{onslide/.code args={<#1>#2}{%
  \only<#1>{\pgfkeysalso{#2}}
}}

\newcommand{\tikzplaceholder}[1]{%
  \begingroup\shorthandoff{;}\tikz[na] \node[coordinate] (#1) {};\endgroup}

\newcommand{\blueoverlay}[2]{%
  \tikz[baseline]{ \node[fill=blue!20,anchor=base,rounded corners=2pt] (#1) {$#2$}; }}

\usetheme{moulardthesis}

\DeclareUnicodeCharacter{00A0}{~}

\title{Closed-loop trajectory following on the HRP-2 platform using ROS}
\author{Thomas Moulard}
\date{Tuesday, 27 November 2012}


% Setup pdf meta-data.
\hypersetup{
pdfauthor = {Thomas Moulard},
pdftitle = {Closed-loop trajectory following on the HRP-2 platform using ROS},
pdfsubject = {Closed-loop trajectory following on the HRP-2 platform using ROS},
pdfkeywords = {},
pdfcreator = {Thomas Moulard},
pdfproducer = {Thomas Moulard}
}


\begin{document}

{
  \usebackgroundtemplate{%
    \includegraphics[width=\paperwidth,height=\paperheight]{%
      src/slides/demo1.jpg}}
  \begin{frame}[plain]
    \titlepage
  \end{frame}
}

%%%%%%%%%%%%%%%%%%%%%%%%%%%%%%%%%%%%%%%%%%%%%%%%%%%%%%%

\begin{slideAction}
  \frametitle{Past work}
  \begin{center}
  \begin{description}
  \item[2009] Master internship at CNRS-AIST JRL\\
    RobOptim, optimization framework for robotics
  \item[2009-2012] PhD at LAAS-CNRS (Toulouse)\\
    Closed-loop trajectory following for legged robots
  \item[2012-2014] JSPS post-doc at CNRS-AIST JRL\\
    Simulating and reproducing human motions using humanoid robots
  \end{description}
  \end{center}
\end{slideAction}

\fullFrameVideo{src/slides/over.jpg}{vid/humanoids11-lbaudouin3.mp4}{%
    \tiny{%
      \vspace{-.6cm}%
      ~\includegraphics[height=.25cm]{src/slides/idea.pdf}~%
      Real-time Replanning Using 3D Environment for Humanoid Robot,
      Proc. of Humanoids'11.%
    }%
}

\maxFrameImageWidth{src/slides/footsteps1.jpg}

\begin{slideAction}
  \frametitle{Robot control}

  \begin{changeleftmargin}{1.1cm}
  \begin{center}

    \begin{tikzpicture}[%
        >=latex,
        node distance=1em,
        fixed/.style={
          text width=120pt}]
      \node [state,fixed] (com) {center of mass};
      \node [state,fixed, below=of com] (lf) {left foot};
      \node [state,fixed, below=of lf] (rf) {right foot};
      \node [state,fixed, below=of rf] (posture) {posture};

      \node [left=of com] (fort) {high priority};
      \node [left=of posture] (faible) {low priority};
      \draw[->, line width=1.05]
      (fort.south) -- (faible.north);
    \end{tikzpicture}
  \end{center}
  \end{changeleftmargin}
\end{slideAction}

\begin{slideAction}
  \frametitle{Motivation}
  \begin{center}
    \begin{tikzpicture}[>=latex]
        \def\w{0.5}
        \def\h{0.75}

        \foreach \dy in {0., 1., 2., 3.}
                 {
                   % left step planned
                   \filldraw[pattern=north east lines] (0., \dy)
                   rectangle (0. + \w, \dy + \h);

                   % right step planned
                   \filldraw[pattern=north west lines] (1., \dy + 0.5)
                   rectangle (1. + \w, \dy + \h + 0.5);
                 }

        % before correction
        \def\driftx{0.1}
        \def\drifty{0.05}
        \def\drifttheta{5.}

        \def\dy{0}
        % left step planned
        \draw<2->[style=solid, color=red, rotate=\drifttheta * \dy,line width=1.25]
        (0. + \driftx * \dy, \drifty * \dy + \dy)
        rectangle (0. + \driftx * \dy + \w, \drifty * \dy + \dy + \h);

        % right step planned
        \draw<3->[style=solid, color=red, rotate=\drifttheta * \dy + \drifttheta / 2.,line width=1.25]
        (1. + \driftx * \dy + \drifty / 2., \drifty * \dy + \drifty / 2. + \dy + 0.5)
        rectangle (1. + \driftx * \dy + \drifty / 2. + \w, \drifty * \dy + \drifty / 2. + \dy + \h + 0.5);

        \def\dy{1}
        % left step planned
        \draw<4->[style=solid, color=red, rotate=\drifttheta * \dy,line width=1.25]
        (0. + \driftx * \dy, \drifty * \dy + \dy)
        rectangle (0. + \driftx * \dy + \w, \drifty * \dy + \dy + \h);

        % right step planned
        \draw<5->[style=solid, color=red, rotate=\drifttheta * \dy + \drifttheta / 2.,line width=1.25]
        (1. + \driftx * \dy + \drifty / 2., \drifty * \dy + \drifty / 2. + \dy + 0.5)
        rectangle (1. + \driftx * \dy + \drifty / 2. + \w, \drifty * \dy + \drifty / 2. + \dy + \h + 0.5);


        \def\dy{2}
        \draw<6->[style=solid, color=red,line width=1.25, rotate=\drifttheta] (0. + \driftx, \dy + \drifty)
        rectangle (0. + \w + \driftx, \dy + \h + \drifty);

        % --- after correction ---

        % left step planned
        \draw<7->[style=solid, color=green,line width=1.25] (0., \dy)
        rectangle (0. + \w, \dy + \h);

        \uncover<7->{
        \draw[smooth,rounded corners=1ex,<-,thick]
        (-0.3, \dy) -- (-0.3 - 0.05, \dy + 0.2) -- (-0.3 - 0.2, \dy + 0.4)
        node[midway,left]
        {
          $\delta \mathbf{x}$
        };
        }


        \draw<7->[style=solid, color=green,line width=1.25] (0., \dy)
        rectangle (0. + \w, \dy + \h);
        \draw<7->[style=solid, color=green,line width=1.25] (1., \dy + 0.5)
        rectangle (1. + \w, \dy + \h + 0.5);
        \def\dy{3}
        \draw<7->[style=solid, color=green,line width=1.25] (0., \dy)
        rectangle (0. + \w, \dy + \h);
        \draw<7->[style=solid, color=green,line width=1.25] (1., \dy + 0.5)
        rectangle (1. + \w, \dy + \h + 0.5);

        % right step planned
        \def\dy{2}
        \draw<8->[style=solid, color=orange,line width=1.25, rotate=\drifttheta * 0.5] (1. + \driftx * .5, \dy + 0.5 + \drifty * 0.5)
        rectangle (1. + \driftx * 0.5 + \w, \dy + \drifty * 0.5 + \h + 0.5);

        \def\dy{3}
        % left step planned
        \draw<9->[style=solid, color=orange,line width=1.25, rotate=\drifttheta * 1.] (0. + \driftx * 1., \dy + \drifty * 1.)
        rectangle (0. + \w + \driftx * 1., \dy + \h + \drifty * 1.);

        % right step planned
        \uncover<10-> {
          \draw[style=solid, color=orange,line width=1.25, rotate=\drifttheta * 1.5] (1. + \driftx * 1.5, \dy + 0.5 + \drifty * 1.5)
        rectangle (1. + \w + \driftx * 1.5, \dy + \h + 0.5 + \drifty * 1.5);
        }

      \end{tikzpicture}
      \end{center}
\end{slideAction}

\begin{slideAction}
  \frametitle{Control scheme}

  \begin{changeleftmargin}{1.1cm}
  \begin{center}
    \begin{tikzpicture}[%
        >=latex]
      \node[state] (gen) {Generation};
      \node[state, below=of gen] (corr) {Correction};
      \node[state, below=of corr, align=left,inner sep=0.1ex,
        shape=rectangle split,
        rectangle split part align=left,
        rectangle split parts=5] (ctrl) {Controller%
        \nodepart{two}\footnotesize{center of mass}
        \nodepart{three}\footnotesize{left foot}
        \nodepart{four}\footnotesize{right foot}
        \nodepart{five}\footnotesize{posture}};
      \node[state,right=2cm of corr] (loc) {localization};

      %\draw[-,style=dashed] (2,-3) -- (2,3);

      \uncover<3-> {
        \draw[->] (loc.west) -- (corr.east)
        node [at start,left,font=\tiny,yshift=0.2cm]
        {$\hat{\mathbf{x}}(t)$};
      }

      \uncover<2-> {
        \draw[->] (gen.south) -- (corr.north)
        node [at start,right,font=\tiny,yshift=-0.2cm]
        {$(\gamma_{\text{feet}}(t), \gamma_{\text{com}}(t))$};
      }

      \uncover<3-> {
        \draw[->] (corr.south) -- (ctrl.north)
        node [at start,right,font=\tiny,yshift=-0.25cm]
        {$(\gamma'_{\text{feet}(t)}, \gamma'_{\text{com}}(t))$};
      }

      \uncover<4-> {
        \draw[->] (ctrl.south) --++ (0,-1)
        node [at start,right,font=\tiny,yshift=-0.25cm]
        {$\dot{\mathbf{q}}(t)$};
      }

      \draw[<-] (gen.north) --++ (0,1) node [left,font=\tiny,yshift=-0.25cm]
           {$\mathbf{S}$};
    \end{tikzpicture}%
    %
    \footnotesize{%
      \hspace{-5cm}%
      \includegraphics[height=.5cm]{src/slides/idea.pdf}~%
      Trajectory Following for Legged Robots, Proc. of Biorob'12.%
    }%
  \end{center}
  \end{changeleftmargin}
\end{slideAction}

\begin{slideAction}
  \frametitle{Correction timing}

  \begin{center}
    \vspace{-.5cm}
    \begin{tikzpicture}
      \begin{axis}[
          grid=major,
          legend style={
            at={(0.5,-0.2)},
            anchor=north,
            legend columns=3,
            cells={anchor=west},
            font=\footnotesize,
            rounded corners=2pt,
            draw=white
          },
          xlabel=time ($s$),
          ylabel=$x$ position ($m$),
          no markers,
          yticklabel pos=lower
          ]

        \addplot[
          Thoughtless, thick,
          dashed
        ] table [
          x expr=(\thisrowno{0}-0)*0.005,
          y expr=\thisrowno{1}+0.04
          ]{dat/feet_follower_feet-follower-com.dat};
        \addlegendentry{center of mass}

        \addplot[
          ThoughtYouWere,
          dashed,
        ] table[
          x index=0,
          y index=4,
          x expr=(\thisrowno{0}-0)*0.005,
          y expr=\thisrowno{4}+0.04
        ] {dat/feet_follower_feet-follower-left-ankle.dat};
        \addlegendentry{left foot}

        \addplot[
          Thoughtless2, thick,
          dashed,
        ] table[
          x index=0,
          y index=4,
          x expr=(\thisrowno{0}-0)*0.005,
          y expr=\thisrowno{4}+0.04
        ] {dat/feet_follower_feet-follower-right-ankle.dat};
        \addlegendentry{right foot}

        \addplot[
          Thoughtless, thick,
        ] table[
          x expr=(\thisrowno{0}-0)*0.005
        ]{dat/feet_follower_correction-com.dat};
        \addlegendentry{center of mass (corr.)}

        \addplot[
          ThoughtYouWere, thick,
        ] table[
          x index=0,
          y index=4,
          x expr=(\thisrowno{0}-0)*0.005,
        ] {dat/feet_follower_correction-left-ankle.dat};
        \addlegendentry{left foot (corr.)}

        \addplot[
          Thoughtless2, thick,
        ] table[
          x index=0,
          y index=4,
          x expr=(\thisrowno{0}-0)*0.005,
        ] {dat/feet_follower_correction-right-ankle.dat};
        \addlegendentry{right foot (corr.)}
      \end{axis}

      \def\w{0.5}
      \def\h{0.75}
      \def\dx{8.}

      \def\dy{.5}
      % left step planned 1
      \filldraw[pattern=north east lines,%
        draw=ThoughtYouWere] (\dx + 0., \dy + 0.5)
      rectangle (\dx + 0. + \w, \dy + \h + 0.5);

      % left step planned 1 - drift
      \uncover<2-> {
        \draw[style=solid,draw=ThoughtYouWere,%
          line width=1.25] (\dx, \dy - 0.3 + 0.5)
        rectangle (\dx + \w, \dy + \h - 0.3 + 0.5);
      }

      \filldraw[pattern=north east lines,%
        draw=Thoughtless2] (\dx + 1., \dy + 0.5)
      rectangle (\dx + 1. + \w, \dy + \h + 0.5);
      \draw<2->[style=solid,color=Thoughtless2,%
        line width=1.25] (\dx + 1., \dy - 0.3 + 0.5)
      rectangle (\dx + \w + 1., \dy + \h - 0.3 + 0.5);


      % right step planned 1
      \filldraw[pattern=north east lines,%
        draw=Thoughtless2] (\dx + 1., \dy + 3.)
      rectangle (\dx + 1. + \w, \dy + \h + 3.)
      node [pos=0.5,anchor=north,font=\tiny,yshift=0.75cm] {step 1};

      % right step planned 1 - corrected
%      \uncover<2-> {
%        \draw[style=solid,draw=Thoughtless2,line width=1.25]
%        (\dx + 1., \dy + 3.-0.3)
%        rectangle (\dx + 1. + \w, \dy + \h + 3. -0.3);
%      }

      \uncover<3-> {
        \draw[style=solid,draw=Thoughtless2,%
          line width=1.25] (\dx + 1., \dy + 3.)
        rectangle (\dx + 1. + \w, \dy + \h + 3.);
      }

      \def\dy{3.}
      % left step planned 2
      \filldraw[pattern=north east lines,%
        draw=ThoughtYouWere] (\dx + 0., \dy + 0.5)
      rectangle (\dx + 0. + \w, \dy + \h + 0.5)
      node [pos=0.5,anchor=north,font=\tiny,yshift=0.75cm] {step 2};

      % left step planned 2 - corrected
      \uncover<4-> {
        \draw[style=solid,draw=ThoughtYouWere,%
          line width=1.25] (\dx + 0., \dy + 0.5)
      rectangle (\dx + 0. + \w, \dy + \h + 0.5);
      }
    \end{tikzpicture}
  \end{center}
\end{slideAction}

\begin{slideAction}
  \frametitle{Center of mass trajectory update}

  \begin{changeleftmargin}{1.1cm}
    \begin{itemize}
    \item Use the linear model.
    \item Suited to a local, small adaptation.
    \item Analytical real time solution.
    \end{itemize}

    \begin{equation*}
      \mathbf{z} = \mathbf{c} - \frac{z_c}{g} \ddot{\mathbf{c}}
    \end{equation*}

%    \begin{equation*}
%      \mathbf{c}^{(n)}(t) = \mathbf{c}(t)  + \sum_{i=1}^n \Delta^{t_i}(t)
%    \end{equation*}
  \end{changeleftmargin}
\end{slideAction}

\fullFrameVideo{src/slides/demo.jpg}{vid/biorob12.mp4}{}


\begin{slideDecision}
  \frametitle{ROS for HRP-2}

  \begin{changemargin}{-1cm}{-1cm}
  \begin{itemize}
  \item ROS has been used to develop recent demonstrations.
  \item Used for communication with controller, vision, diagnostic, etc.
  \item What I mean by ROS:
    \begin{itemize}
    \item The robotics middleware,
    \item and the set of tools related to it.
    \end{itemize}
  \end{itemize}
  \end{changemargin}
\end{slideDecision}

\fullFrameVideo{src/slides/demo2.jpg}{vid/13icra.mp4}{%
    \tiny{%
      \vspace{-.6cm}%
      ~\includegraphics[height=.25cm]{src/slides/idea.pdf}~%
      Reliable Indoor Navigation on Humanoid Robots using
      Vision-Based Localization, submitted to ICRA'13.%
    }%
}

\maxFrameImageWidth{src/slides/archi.pdf}

\begin{slideDecision}
  \frametitle{Architecture overview}

  Developped at LAAS:
  \begin{itemize}
  \item HRP-2 14 URDF robot model,
  \item hrpsys and view simulation bridge,
  \item dynamic-graph bridge,
  \item localization component.
  \end{itemize}

  Re-used components:
  \begin{itemize}
    \item Vision pipeline% (calibration, low-level image processing).
    \item Stereo vision% (disparity, point cloud generatin and filtering).
    \item Coordinate frame management system (tf).
    \item GUI, diagnostic tools.
  \end{itemize}
\end{slideDecision}

\maxFrameImageWidth{src/slides/footsteps1.jpg}

\begin{slideDecision}
  \frametitle{Locomotion}

  \begin{itemize}
  \item Locomotion data can be displayed on advance.
  \item Any ROS compatible GUI can be used (rviz here).
  \item HRP-2 URDF model is used for display.
  \end{itemize}
\end{slideDecision}

\maxFrameImage{src/slides/hrp2_urdf.jpg}

\begin{slideDecision}
  \frametitle{Geometrical frames handling}

  \begin{itemize}
  \item tf is used to store frames.
  \item Ensure data consistency between various components.
  \item Provide a clean way of dealing with delays and time synchronization.
  \item REP 120: Coordinate Frames for Humanoids Robots.
  \end{itemize}
\end{slideDecision}

\maxFrameImageWidth{src/slides/footsteps2.jpg}

\begin{slideDecision}
  \frametitle{Vision}

  \begin{itemize}
  \item hrpsys bridge provides encoders values and send their relative
    position to tf.
  \item The vision pipeline integrates debayerisation,
    rectification, disparity/point cloud from stereo pairs.
  \item With HRP-2 URDF and the frame management system, reprojection
    on rectified images is possible.
  \item All components are always running, only necessary computation
    are done.
  \item Ability to regroup several components into one when needed
    (nodelets).
  \end{itemize}
\end{slideDecision}

\maxFrameImageHeight{src/slides/calibextrinsic.jpg}
\maxFrameImageHeight{src/slides/stereo1.jpg}
\maxFrameImageWidth{src/slides/shelf.png}

\begin{slideDecision}
  \frametitle{Object tracking in ROS}

  \begin{itemize}
  \item Difficult piece of software: need calibration, model, etc.
  \item Now relies on the vision pipeline, works as a nodelet.
  \item Now easier to use, more external users (PAL~robotics).
  \end{itemize}
\end{slideDecision}

\maxFrameImageWidth{src/slides/rviz-full.jpg}

\begin{slideDecision}
  \frametitle{Other tools}

  \begin{itemize}
  \item Motion capture component.
  \item Diagnostic board.
  \item Unified log/console.
  \end{itemize}
\end{slideDecision}

\begin{slideDecision}
  \frametitle{Conclusion}

  \begin{itemize}
  \item A framework for precise trajectory following is proposed.
  \item Experiments realized with motion capture and visual odometry.
  \item Requires a more complex robotics infrastructure.
  \item ROS proved itself useful to achieve this goal.
  \end{itemize}
\end{slideDecision}


\maxFrameImage{src/slides/hrp2-thx.jpg}
\end{document}
